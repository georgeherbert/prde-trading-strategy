\documentclass[conference]{IEEEtran}
% \usepackage{cite}
\usepackage{amsmath,amssymb,amsfonts}
% \usepackage{algorithmic}
\usepackage{graphicx}
\usepackage{multirow}
% \usepackage{textcomp}
% \usepackage{xcolor}
\graphicspath{{./images/}}
\def\BibTeX{{\rm B\kern-.05em{\sc i\kern-.025em b}\kern-.08em
    T\kern-.1667em\lower.7ex\hbox{E}\kern-.125emX}}
\begin{document}

\title{Evaluation of the Parameterized-Response Differential Evolution Trader-Agent}

\author{\IEEEauthorblockN{George Herbert}
\IEEEauthorblockA{\textit{Department of Computer Science} \\
\textit{University of Bristol}\\
Bristol, United Kingdom \\
cj19328@bristol.ac.uk}
}

\maketitle

\begin{abstract}
This paper evaluates the Paramaterized-Response Differential Evolution (PRDE) trader-agent.
\end{abstract}

\begin{IEEEkeywords}
Automated Trading, Financial Markets, Adaptive Trader-Agents
\end{IEEEkeywords}

\section{Introduction}

Automated trading accounts for an unprecedented amount of activity in modern financial markets.
These algorithms have the ability to execute trades at a frequency simply unachievable for humans beings; thus, the behaviour of many of these markets has fundamentally shifted.
Notably, the `flash crash' in US financial markets on 6 May 2010 has been partially attributed to high-frequency trading algorithms aggressively reselling short-term positions to one another.
Many automated trading algorithms have the additional quality of being adaptive: they adjust their strategy to extract maximum profit from the market in which they are operating.
Cliff \cite{PRDE} described these contemporary markets---in which competing adaptive algorithms are simultaneously engaged in coninual adjustment to maintain profitability---as co-evolutionary systems.

Large amounts of research has been conducted to understand the dynamics of these markets.
Recently, Cliff introduced the Parameterized-Response Zero-Intelligence (PRZI) \cite{PRZI} trader: a nonadaptive generalisation of the ZIC \cite{GodeSunder} trader---the difference lies in the probability mass function (PMF) used to generate quote prices.
Each individual ZIC trader samples their quote price from a fixed uniform distribution, whereas each PRZI trader is governed by a strategy parameter $s\in[-1, 1]\in\mathbb{R}$ that determines the PMF the trader samples from; the shape of this PMF determines how `urgent' or `relaxed' the trader acts.
As $s\to1$ the distribution is evermore biased towards `urgent' quote prices---those closest to the least profitable price for the trader, but most likely to attract a willing counterparty---conversely, as $s\to-1$, the distribution is biased towards `relaxed' quote prices---those that generate the most profit for the trader, but are considerably less likely to attract a counterparty.
When $s=0$, the PMF is uniform, identical to that of a ZIC trader.

PRZI Stochastic Hillclimber (PRSH) \cite{PRSH} is an extension to the PRZI automated-trader algorithm, also introduced by Cliff.
The strategy parameter $s$ is dynamically altered by the algorithm in an attempt to increase profitability.
Each PRSH trader maintains a private local population $\mathcal{K}$ of $k$ strategy parameters; each of which it evaluates for a specific period of time via a loop to identify which is most profitable.
The most profitable strategy $s_0$ is `mutated' $k-1$ times---these $k$ values comprise the new elements of set $\mathcal{K}$.

PRZI Differential Evolution (PRDE) \cite{PRDE} is further extension of the PRZI algorithm, and a successor to PRSH; it is the most recent algorithm published by Cliff in the PRZI `family' of trader-agents.
It replaces the simple stochastic hill-climber with a differential evolution (DE) optimisation system \cite{StornPrice}.
Each PRDE trader maintains its own DE system with a population of candidate $s$-values of size $\mathrm{NP}\ge4$, which for trader $i$ can be denoted by $s_{i,1},s_{i,2},...,s_{i,\mathrm{NP}}$.
Once a particular strategy $s_{i,x}$ has been evaluated, three other distinct $s$-value are chosen at random from the population maintained by trader $i$: $s_{i,a}$, $s_{i,b}$ and $s_{i,c}$ such that $x\ne a\ne b\ne c$.
A new candidate strategy $s_{i,y}$ is constructed as follows:
\[
s_{i,y}=\max(\min(s_{i,a}+F_i(s_{i,b}-s_{i,c}),1), -1)
\]
where $F_i$ is the trader's differential weight coefficient.
This is very similar to the standard DE/rand/1 algorithm, with addition of $\max$ and $\min$ functions to constrain the output $s_{i,y}\in[-1,1]$.
The fitness of $s_{i,y}$ is evaluated and if it performs better than $s_{i,x}$ then $s_{i,y}$ replaces $s_{i,x}$; otherwise, it is discarded and then the next strategy is evaluated.
Cliff made one additional modification to his implementation of DE/rand/1 algorithm in PRDE to deal with convergence issues that arised from $s_{i,b}-s_{i,c}$ tending very close to zero.
He introduced a simple vector-perbutation mechanism: if at any time the standard deviation of the candidate $s$-values in trader $i$'s private population is less than $0.0001$, then a randomly selected candidate is provided a value drawn at random from the uniform distribution $U(-1,1)$.

Cliff's conducted experiments on the \textit{Bristol Stock Exchange} (BSE) (see \cite{BSE, BSEPaper}) to analyse and evaluate PRSH and PRDE.
BSE is a freely-available, open-source simulation a LOB-based financial exchange.
Notably, Cliff identified that markets populated by PRDE traders were approximately 100\% more economically efficient than those populated by PRSH traders \cite{PRDE}.
However, the PRDE traders implemented in Cliff's experiments did not deviate from a differential weight of $F=0.8$ and a number in population of $\mathrm{NP}=4$ (i.e. the minimum viable value).
He stressed the importance of future work to explore the effects on the market's dynamics of altering these key parameters.
Cliff proposed two lines of future research regarding exploration of different values for the differential weight coefficient and the number in population.
He proposed an exploration into the effects on the market's dynamics of altering the two key parameters homogeneously---with all traders maintaining the same values of $F$ and $\mathrm{NP}$.
He also proposed an arguably more intriguing exploration into effects that arise from altering the two key parameters heterogeneously---with different traders being provided different values of $F$ and $\mathrm{NP}$.

\section{Homogeneous Exploration of $F$ and $\mathrm{NP}$}

My initial experiments in this paper focus on the effect of altering the number in population $\mathrm{NP}$.
To do so, I designed a set of experiments with a similar setup as Cliff \cite{PRDE}.
In each experiment, I implemented a homogeneous population of $N_T=30$ PRDE traders with an equal number of buyers and sellers.
The population was homogenous with respect to the fact that each trader had an equal differential weight coefficient of $F=0.8$, and an equal number in population $\mathrm{NP}$, which I altered for each experiment.
Each experiment had perfect elasticity of supply and demand: all buyers had a limit price of \$140 per unit, whilst all sellers had a limit price of \$60.
After two traders engaged in a trade, they were rendered inactive until their stock was replenished, which occurred approximately every five seconds.
I ran each experiment for 50 simulated days.
I used profit per unit time as a measure of market efficiency; more specifically, I used profit per second (PPS).
The PPS was computed every $\Delta t=3600$ seconds, by dividing the total profit a given trader achieved during the prior $\Delta t$ seconds by $\Delta t$.



\section{Heterogeneous Exploration of $F$ and $\mathrm{NP}$}

\section{Extending PRDE}

\bibliographystyle{IEEEtran}
\bibliography{refs}

\end{document}
