\documentclass[conference]{IEEEtran}
% \usepackage{cite}
\usepackage{amsmath,amssymb,amsfonts}
% \usepackage{algorithmic}
\usepackage{graphicx}
\usepackage{multirow}
% \usepackage{textcomp}
% \usepackage{xcolor}
\graphicspath{{./images/}}
\def\BibTeX{{\rm B\kern-.05em{\sc i\kern-.025em b}\kern-.08em
    T\kern-.1667em\lower.7ex\hbox{E}\kern-.125emX}}
\begin{document}

\title{Evaluation of the Parameterized-Response Differential Evolution Trader-Agent}

\author{\IEEEauthorblockN{George Herbert}
\IEEEauthorblockA{\textit{Department of Computer Science} \\
\textit{University of Bristol}\\
Bristol, United Kingdom \\
cj19328@bristol.ac.uk}
}

\maketitle

\begin{abstract}
This paper evaluates the Paramaterized-Response Differential Evolution (PRDE) trader-agent.
\end{abstract}

\begin{IEEEkeywords}
Automated Trading, Financial Markets, Adaptive Trader-Agents
\end{IEEEkeywords}

\section{Introduction}

\subsection{Trader-Agents}

Parameterized-Response Zero Intelligence (PRZI) traders \cite{CliffMetapopulation} are a nonadaptive generalisation of ZIC \cite{GodeSunder} traders---the difference lies in the probability mass function (PMF) used to generate quote prices.
Each individual ZIC trader samples their quote price from a fixed uniform distribution, whereas each PRZI trader is governed by a strategy parameter $s\in[-1, 1]\in\mathbb{R}$ that determines the PMF the trader samples from; the shape of this PMF determines how `urgent' or `relaxed' the trader acts.
As $s\to1$ the distribution is evermore biased towards `urgent' quote prices---those closest to the least profitable price for the trader, but most likely to attract a willing counterparty---conversely, as $s\to-1$, the distribution is biased towards `relaxed' quote prices---those that generate the most profit for the trader, but are considerably less likely to attract a counterparty.
When $s=0$, the PMF is uniform, identical to that of a ZIC trader.

PRZI Stochastic Hillclimber (PRSH) \cite{CliffMetapopulation} is an extension to the PRZI automated-trader algorithm.
The strategy parameter $s$ is dynamically altered by the algorithm in an attempt to increase profitability.
Each PRSH trader maintains a private local population $\mathcal{K}$ of $k$ strategy parameters; each of which it evaluates for a specific period of time via a loop to identify which is most profitable.
The most profitable strategy $s_0$ is `mutated' $k-1$ times---these $k$ values comprise the new elements of set $\mathcal{K}$.

Parameterized-Response Zero Intelligence (PRDE) is further extension of the PRZI algorithm, and a successor to PRSH; it replaces the simple stochastic hill-climber with a differential evolution (DE) optimisation system \cite{StornPrice}.
Each PRDE trader maintains its own DE system with a population of candidate $s$-values of size $\mathrm{NP}\ge4$.

\subsection{Differential Evolution}

In a DE system there are a population of candidate solutions; in PRDE, each of these candidate solutions is a stratgey parameter $s\in[-1,1]$.

\section{Experiments}

\cite{BSE} \cite{BSEPaper}

\bibliographystyle{IEEEtran}
\bibliography{refs}

\end{document}
